\documentclass[letterpaper,10pt]{article}
\usepackage[top=2cm, bottom=1.5cm, left=1cm, right=1cm]{geometry}
\usepackage{amsmath, amssymb, amsthm,graphicx}
\usepackage{fancyhdr}
\pagestyle{fancy}

\lhead{\today}
\chead{Homework 4a}
\rhead{Justin Hood}

\newcommand{\Z}{\mathbb{Z}}
\newcommand{\Q}{\mathbb{Q}}
\newcommand{\R}{\mathbb{R}}
\newcommand{\C}{\mathbb{C}}
\newtheorem{lem}{Lemma}

\begin{document}
We shall perform regression on the ``Crest" data based on the model form,
\[y=\beta_0+\beta_1x_1+\beta_2x_2+\beta_3x_3+\epsilon\]
\begin{enumerate}
\item From R, we compute the least squares estimates as,
\begin{align*}
\beta_0 &=30625.907\\
\beta_1 &=3.893\\
\beta_2 &=-29607.315\\
\beta_3 &=86.519
\end{align*}
\item SSE, $s^2$ and $s$
\begin{enumerate}
\item We compute SSE, the sum of squared error, as
\[SSE=894568667.411638\]
\item In order to compute both $s$ and $s^2$, we must compute $k$. $k$ is equal to the subscript of the highest $\beta_i$. As such our model has $k=3$. Then, noting that $n=13$, we compute,
\[s^2=\frac{SSE}{n-(k+1)}=\frac{894568667.411638}{13-(3+1)}=\frac{894568667.411638}{9}\approx 99396519\]
\item We may then compute,
\[s=\sqrt{s^2}\approx 9969.78\]
\end{enumerate}
\item Variation
\begin{enumerate}
\item We compute, $Total Variation = \sum(y_i-\bar{y})^2$. Then,
\[T.V.\approx 21050596826.9231\]
\item Next, we compute $ExplainedVariation = \sum(\hat{y}_i-\bar{y})^2$. Then,
\[E.V.\approx 20156028159.5114\]
\item Finally, we compute $UnexplainedVariation = \sum(y_i-\hat{y}_i)^2$. Then,
\[U.V.\approx 894568667.4116\]
\end{enumerate}
\item Using R, we compute $F(model)=67.59$ with a $p-val=1.709e-6$. Thus, with $\alpha=.05$, Thus, we reject the null hypothesis that all of the $\beta_i=0$. Then, our alternative that $\beta_i\neq 0$ becomes plausible.
\item Using the generated $t$-statistics and associated $p$-values, we shall test each of the computed $\beta_i$ using the null hypotheses $H_0:\ \beta_i=0$ versus the alternative $H_A:\ \beta_i\neq 0$ with $\alpha=.05$. The results follow,\\
\begin{center}
\begin{tabular}{c|c|c|c|l}
Variable & Beta Value & t-value & p-value & Significance\\\hline
Intercept & $\beta_0$ & 1.546 & 0.15647 & \text{Not Significant at }$\alpha=.05$ \\
Budget & $\beta_1$ & 1.871 & .09420 & \text{Not Significant at }$\alpha=.05$ \\
Ratio & $\beta_2$ & -1.243 & .24533 & \text{Not Significant at }$\alpha=.05$ \\
Income & $\beta_3$ & 4.628 & .00124 & \text{Significant at }$\alpha=.05$
\end{tabular}
\end{center}
Similarly, we make the same comparison with $\alpha=.01$, and see that, again the income variable is the only one sigificant at this new level.
\item We shall now compute 90\% confidence intervals for each $\beta_i$. The equation for our interval is,
\[\bigg[b_j\pm t_{\alpha/2}^{n-(k+1)}s_{b_j}\bigg]\]
Using our previously computed $k$ and $n$, we find
\[t^9_{.05}=1.833\]
Then, we compute the intervals as follows:
\begin{center}
\begin{tabular}{c|c|c|c}
$\beta_j$ & $b_j$ & $s_{b_j}$ & Interval \\\hline
$\beta_0$ & 30625.907 & 19808.009 & [-5682.173, 66933.988]\\
$\beta_1$ & 3.893 & 2.081 & [.078527, 7.707473]\\
$\beta_2$ & -29607.315 & 23822.087 & [-73273.20047, 14058.57047]\\
$\beta_3$ & 86.519 & 18.693 & [52.254731, 120.783269]
\end{tabular}
\end{center}
\item Next, we note that our equation of,
\[R^2=\frac{Explained\ Variation}{Total\ Variation}\]
Thus,
\[R^2\approx .9575039\]
To compute the adjusted $R^2$ value, we use the equation,
\[\bar{R}^2=\bigg(R^2-\frac{k}{n-1}\bigg)\bigg(\frac{n-1}{n-(k+1)}\bigg)\]
So,
\[\bar{R}^2=\bigg(0.9575-\frac{3}{13-1}\bigg)\bigg(\frac{13-1}{13-(3+1)}\bigg)\approx 0.9433385\]
\item We now consider the computation of a new point prediction and interval based on the new data of $Budget=25,000$, $Ratio=1.55$ and $Income=1821.70$. Using our computed linear model and $R$, we compute,
\[\hat{y}=239675.4\]
\[P.I.=[209507.3,269843.5]\]
\newpage
\center{R-Code}
\begin{verbatim}
> setwd("~/Desktop/PSM/Fall 2018/Regression/HW4")
> crest=read.table('crest.txt',header = TRUE)
> mod<-lm(Sales~Budget+Ratio+Income, data=crest)
> summary(mod)

Call:
lm(formula = Sales ~ Budget + Ratio + Income, data = crest)

Residuals:
   Min     1Q Median     3Q    Max 
-24447  -2561   1195   3885   9527 

Coefficients:
              Estimate Std. Error t value Pr(>|t|)   
(Intercept)  30625.907  19808.009   1.546  0.15647   
Budget           3.893      2.081   1.871  0.09420 . 
Ratio       -29607.315  23822.087  -1.243  0.24533   
Income          86.519     18.693   4.628  0.00124 **
---
Signif. codes:  0 ‘***’ 0.001 ‘**’ 0.01 ‘*’ 0.05 ‘.’ 0.1 ‘ ’ 1

Residual standard error: 9970 on 9 degrees of freedom
Multiple R-squared:  0.9575,	Adjusted R-squared:  0.9433 
F-statistic: 67.59 on 3 and 9 DF,  p-value: 1.709e-06

> test=data.frame(Budget=16300, Ratio=1.25, Income=547.9)
> predict(mod,test)
       1 
104479.3 
> 105000-104479.3
[1] 520.7
> mod$residuals
          1           2           3           4           5 
   520.6815   1195.3334   9527.2175  -2708.3077  -6270.0669 
          6           7           8           9          10 
  7144.6314   3885.3353 -24446.9075    868.6300   2758.4026 
         11          12          13 
  1446.1279   8639.8221  -2560.8997 
> SSE=sum(mod$residuals^2)
> s2=SSE/(13-(3+1))
> s=sqrt(s2)
> s2
[1] 99396519
> s
[1] 9969.78
> print(s2)
[1] 99396519
> totvar=sum(crest$Sales-mean(crest$Sales))
> mean(crest$Sales)
[1] 148723.1
> diff=crest$Sales-mean(crest$Sales)
> diff
 [1] -43723.077 -43723.077 -27123.077 -34973.077 -34973.077
 [6] -19798.077  -6223.077 -22723.077  13276.923  42901.923
[11]  40276.923  61276.923  75526.923
> diff=diff^2
> sum(diff)
[1] 21050596827
> totvar=sum((crest$Sales-mean(crest$Sales))^2)
> totvar
[1] 21050596827
> X=data.frame(crest[3:5])
> test
  Budget Ratio Income
1  16300  1.25  547.9
> X
   Budget Ratio Income
1   16300  1.25  547.9
2   15800  1.34  593.4
3   16000  1.22  638.9
4   14200  1.00  695.3
5   15000  1.15  751.8
6   14000  1.13  810.3
7   15400  1.05  914.5
8   18250  1.27  998.3
9   17300  1.07 1096.1
10  23000  1.17 1194.4
11  19300  1.07 1311.5
12  23056  1.54 1462.9
13  26000  1.59 1641.7
> yhat=predict(mod,X)
> yhat
       1        2        3        4        5        6        7 
104479.3 103804.7 112072.8 116458.3 120020.1 121780.4 138614.7 
       8        9       10       11       12       13 
150446.9 161131.4 188866.6 187553.9 201360.2 226810.9 
> expvar=sum((yhat-mean(crest$Sales))^2)
> unexvar=sum((crest$Sales-yhat)^2)
> totvar-expvar-unexvar
[1] -4.529953e-06
> expvar
[1] 20156028160
> unexvar
[1] 894568667
> anova(mod)
Analysis of Variance Table

Response: Sales
          Df     Sum Sq    Mean Sq F value    Pr(>F)    
Budget     1 1.7008e+10 1.7008e+10 171.117 3.681e-07 ***
Ratio      1 1.0184e+09 1.0184e+09  10.246   0.01081 *  
Income     1 2.1292e+09 2.1292e+09  21.421   0.00124 ** 
Residuals  9 8.9457e+08 9.9397e+07                      
---
Signif. codes:  0 ‘***’ 0.001 ‘**’ 0.01 ‘*’ 0.05 ‘.’ 0.1 ‘ ’ 1
> fmodel=(expvar/3)/(unexvar/(13-(3+1)))
> summary(mod)

Call:
lm(formula = Sales ~ Budget + Ratio + Income, data = crest)

Residuals:
   Min     1Q Median     3Q    Max 
-24447  -2561   1195   3885   9527 

Coefficients:
              Estimate Std. Error t value Pr(>|t|)   
(Intercept)  30625.907  19808.009   1.546  0.15647   
Budget           3.893      2.081   1.871  0.09420 . 
Ratio       -29607.315  23822.087  -1.243  0.24533   
Income          86.519     18.693   4.628  0.00124 **
---
Signif. codes:  
0 ‘***’ 0.001 ‘**’ 0.01 ‘*’ 0.05 ‘.’ 0.1 ‘ ’ 1

Residual standard error: 9970 on 9 degrees of freedom
Multiple R-squared:  0.9575,	Adjusted R-squared:  0.9433 
F-statistic: 67.59 on 3 and 9 DF,  p-value: 1.709e-06

> expvar/totvar
[1] 0.9575039
> (expvar/totvar-3/12)*(12/9)
[1] 0.9433385
> topred=data.frame(Budget=25000, Ratio=1.55, Income=1821.70)
> predict(mod, topred)
       1 
239675.4 
> predict.lm(mod, topred)
       1 
239675.4 
> predict(mod, topred, interval = "prediction")
       fit      lwr      upr
1 239675.4 209507.3 269843.5
\end{verbatim}
\end{enumerate}
\end{document}

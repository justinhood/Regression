\documentclass[letterpaper,10pt]{article}
\usepackage[top=2cm, bottom=1.5cm, left=1cm, right=1cm]{geometry}
\usepackage{amsmath, amssymb, amsthm,graphicx}
\usepackage{fancyhdr}
\pagestyle{fancy}

\lhead{\today}
\chead{Regression Homework 7}
\rhead{Justin Hood}

\newcommand{\Z}{\mathbb{Z}}
\newcommand{\Q}{\mathbb{Q}}
\newcommand{\R}{\mathbb{R}}
\newcommand{\C}{\mathbb{C}}
\newtheorem{lem}{Lemma}

\begin{document}
\begin{enumerate}
\item Find the four seasonal factors for quarters 1,2,3,4.\\\\
To compute the seasonal factors, we first note that our model takes the form
\[y_t=TR_t+SN_t+CL_t+IR_t\]
Using centered moving averages, we may compute our estimate
\[sn_t+ir_t=y-CMA_t\]
We then average by season, and normalize the values as,
\[sn_t=\bar{sn}_t-\bigg(\sum_{t=1}^L\bar{sn}_t/L\bigg)\]
Our seasonal factors are then computed as
\begin{align*}
sn_{q1}&= 70.59375\\
sn_{q2}&= 210.76042\\
sn_{q3}&= -76.94792\\
sn_{q4}&= -204.40625
\end{align*}
\item We now predict the values of sales in the next year by the formula $y=tr+sn$ 
These values are as follows
\begin{align*}
t_{17}\Rightarrow Q_1 &= 621.3292\\
t_{18}\Rightarrow Q_2 &= 781.2515\\
t_{19}\Rightarrow Q_3 &= 513.2988\\
t_{20}\Rightarrow Q_4 &= 405.5961
\end{align*}
\item Finally, we compute the prediction intervals for the first two quarters using the $B$ value from the text. These intervals are then,
\[t_{17}:[535.69235,\ 706.96598\]
\[t_{18}:[695.49893,\ 867.00401]\]
\end{enumerate}
\end{document}
